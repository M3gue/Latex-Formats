\section{Matemáticas}

\subsection{Modo matemático en línea}

Las ecuaciones en línea son las que hacen parte de un párrafo, insertándose entre texto normal. Para escribir en modo matemático dentro de una línea debemos poner lo deseado entre los signos \verb|\(| y \verb|\)|. Antiguamente se hacía lo mismo con los signos \verb|$|. Actualmente se puede seguir haciendo, pero lo correcto es hacer uso de los signos \verb|\(| y \verb|\)|.

\textbf{Ejemplo 1}

Texto Texto Texto \(y=mx+b\) Texto Texto Texto 

\begin{myquote}
	\begin{lstlisting}
Texto Texto Texto \(y=mx+b\) Texto Texto Texto 
	\end{lstlisting}
\end{myquote}

\textbf{Ejemplo 2}

Texto Texto Texto $y=mx+b$ Texto Texto Texto 

\begin{myquote}
	\begin{lstlisting}
Texto Texto Texto $y=mx+b$ Texto Texto Texto 
	\end{lstlisting}
\end{myquote}

\subsection{Modo matemático entre párrafos}

Los signos \verb|\[| y \verb|\]|, y \verb|$$| nos permiten entrar en modo matemático, pero entre párrafo, centrando lo que se encuentre dentro de los mismos: 

\textbf{Ejemplo 1}

Texto Texto Texto \[y=mx+b\] Texto Texto Texto 

\begin{myquote}
	\begin{lstlisting}
	Texto Texto Texto \[y=mx+b\] Texto Texto Texto 
	\end{lstlisting}
\end{myquote}

\textbf{Ejemplo 2}

Texto Texto Texto $$y=mx+b$$ Texto Texto Texto 

\begin{myquote}
	\begin{lstlisting}
	Texto Texto Texto $$y=mx+b$$ Texto Texto Texto 
	\end{lstlisting}
\end{myquote}

\subsubsection{Entorno \texttt{equation}}

Este entorno nos permite enumerar la ecuación que insertemos. Esta quedará entre párrafos y estará centrada. A la derecha de ecuación nos pondrá del número de esta. Es importante tener en cuenta que este entorno solo nos permite escribir una ecuación a la vez.

\textbf{Ejemplo 1}

\begin{equation}
	y=mx+b
\end{equation}
\begin{equation}
	a=bh
\end{equation}

\begin{myquote}
	\begin{lstlisting}
\begin{equation}
	y=mx+b
\end{equation}

\begin{equation}
	a=bh
\end{equation}
	\end{lstlisting}
\end{myquote}

\subsection{Escritura matemática}

\subsubsection{Símbolos}

Algunos símbolos en Latex deben ser escritos de forma especial, mientras que otros se comportan como letras, por lo que solo deben ser escritos normalmente. Veamos a continuación algunas tablas con los símbolos más comunes, aunque si se quiere ver más símbolos, recomiendo dirigirse al siguiente link: \href{https://rinconmatematico.com/instructivolatex/formulas.htm}{\textcolor{light-blue}{RinconMatematico}}

\begin{center}
	\begin{tabular}{|c|}
		\hline
		Simbolos directos\\
		\hline
		+ \hspace{30pt} - \hspace{30pt} = \hspace{30pt} ! \hspace{30pt} /\\
		( \hspace{30pt} ) \hspace{30pt} [ \hspace{30pt} ] \hspace{30pt} <\\
		> \hspace{30pt} | \hspace{30pt} ' \hspace{30pt} : \hspace{30pt} *\\
		\hline	
	\end{tabular}
\end{center}

\begin{tabularx}{\textwidth}{|c|>{\ttfamily}X|c|>{\ttfamily}X|}
	\hline
	\multicolumn{4}{|c|}{Simbolos indirectos}                                                             \\ \hline
	    Simbolo     & \multicolumn{1}{l|}{Comando} &       Simbolo       & \multicolumn{1}{l|}{Comando} \\ \hline
	    \(\pm\)     & \verb|\pm|                    &       \(\mp\)       & \verb|\mp|                    \\ \hline
	  \(\times\)    & \verb|\times|                 &      \(\neq\)       & \verb|\neq|                   \\ \hline
	   \(\neg\)     & \verb|\neg|                   &     \(\infty\)      & \verb|\infty|                 \\ \hline
	  \(\bigcap\)   & \verb|\bigcap|                &     \(\bigcup\)     & \verb|\bigcup|                \\ \hline
	 \(\bigwedge\)  & \verb|\bigwedge|              &     \(\bigvee\)     & \verb|\bigvee|                \\ \hline
	\(\rightarrow\) & \verb|\rightarrow|            & \(\leftrightarrow\) & \verb|\leftrightarrow|        \\ \hline
	\(\Rightarrow\) & \verb|\Rightarrow|            & \(\Leftrightarrow\) & \verb|\Leftrightarrow|        \\ \hline
	  \(\approx\)   & \verb|\approx|                &     \(\equiv\)      & \verb|\equiv|                 \\ \hline
	   \(\leq\)     & \verb|\leq|                   &      \(\geq\)       & \verb|\geq|                   \\ \hline
	  \(\angle\)    & \verb|\angle|                 &     \(\vec{x}\)     & \verb|\vec{}|                 \\ \hline
	  \(\forall\)   & \verb|\forall|                &       \(\in\)       & \verb|\in|                    \\ \hline
	 \(\exists \)   & \verb|\exists|                &        \( \)        & \verb||                       \\ \hline
\end{tabularx}

\subsubsection{Espacios}

Para insertar un espacio cuando se está en modo matemático se deben hacer uso de los siguientes comandos:

\begin{center}
	\begin{tabular}{|>{\ttfamily}c|c|}
		\hline
		\textrm{Comando} & Ejemplo\\
		\hline	
		\textrm{Normal} & $y=mx+b$\\
		\hline	
		\verb|\;| & $y\;=\;m\;x\;+\;b$\\
		\hline	
		\verb|\:| & $y\:=\:m\:x\:+\:b$\\
		\hline	
		\verb|\,| & $y\,=\,m\,x\,+\,b$\\
		\hline
	\end{tabular}
\end{center}

\subsection{Paquete \texttt{amsmath}}

El paquete \texttt{amsmath} nos permite ampliar las opciones, los entornos y soluciona algunos errores del modo matemático que tiene Latex por defecto. Además agrega símbolos adicionales para usar en los entornos matemáticos. Estos se pueden consultar en la siguiente página: \href{http://milde.users.sourceforge.net/LUCR/Math/mathpackages/amsmath-symbols.pdf}{\textcolor{light-blue}{Günter Milde}}


Veamos algunas de las opciones que agrega este paquete:

\subsubsection{Entorno \texttt{equation*}}

Este entorno es igual que el entorno \texttt{equation}, con la única diferencia de que este encorno no numera las ecuaciones.

\textbf{Ejemplo 1}

\begin{equation*}
y=mx+b
\end{equation*}
\begin{equation*}
a=bh
\end{equation*}

\begin{myquote}
	\begin{lstlisting}
\begin{equation*}
	y=mx+b
\end{equation*}
	
\begin{equation*}
	a=bh
\end{equation*}
	\end{lstlisting}
\end{myquote}


\subsubsection{Entorno \texttt{split}}

Este entorno nos permite alinear diferentes ecuaciones. Este entorno debe de hacerse dentro del entorno \texttt{equation} o \texttt{equation*} para funcionar. Si se selecciona el entorno \texttt{equation}, solo se va a asignar un número para todas las ecuaciones que se hagan. Este entorno es recomendado para los casos en los que se quiere despejar una ecuación.

Este entono tiene una disposición similar a la de una tabla. Los saltos de línea de deben hacer con un \verb|\\|, y la ecuación debe separarse por \verb|&|, que sirve para establecer el punto donde una ecuación debe ser alineada. El signo que vaya después de \verb|&| será el punto donde las ecuaciones se alineen:

\textbf{Ejemplo 1}

\begin{equation}
	\begin{split}
 		x+4-y+10+2y &= 2x\\
 		14-y+2y &= 3x\\
 		y &= 3x-14\\
	\end{split}
\end{equation}

\begin{myquote}
	\begin{lstlisting}
\begin{equation}
	\begin{split}
		x+4-y+10+2y &= 2x\\
		14-y+2y &= 3x\\
		y &= 3x-14\\
	\end{split}
\end{equation}
	\end{lstlisting}
\end{myquote}

\textbf{Ejemplo 2}

\begin{equation*}
	\begin{split}
		x+4-y+10+ &2y =2x\\
		14-y+ &2y =3x\\
		&y =3x-14\\
	\end{split}
\end{equation*}

\begin{myquote}
	\begin{lstlisting}
\begin{equation*}
	\begin{split}
		x+4-y+10+ &2y =2x\\
		14-y+ &2y =3x\\
		&y =3x-14\\
	\end{split}
\end{equation*}
	\end{lstlisting}
\end{myquote}

\subsubsection{Entorno \texttt{multline}}

Este entorno está pensado para escribir ecuaciones que son tan largas que no pueden estar en una solo línea. Este paquete parte la ecuación en dos partes separadas por un \verb|\\|. La primera parte se alineará a la izquierda y la segunda a la derecha. Si se separa en más de dos partes las partes intermedias se alinearán al centro. De igual manera existe un entorno que enumera y otro que no lo hace, siendo \texttt{multline} y \texttt{multline*} respectivamente. Veamos ejemplos de este entorno:

\textbf{Ejemplo 1}

\begin{multline}
p(x) = 3x^6 + 14x^5y + 590x^4y^2 + 19x^3y^3\\ 
- 12x^2y^4 - 12xy^5 + 2y^6 - a^3b^3
\end{multline}

\begin{myquote}
	\begin{lstlisting}
\begin{multline}
	p(x) = 3x^6 + 14x^5y + 590x^4y^2 + 19x^3y^3\\ 
	- 12x^2y^4 - 12xy^5 + 2y^6 - a^3b^3
\end{multline}
	\end{lstlisting}
\end{myquote}

\textbf{Ejemplo 2}

\begin{multline*}
p(x) = 3x^6 + 14x^5y\\
+ 590x^4y^2 + 19x^3y^3\\
- 12x^2y^4 - 12xy^5\\
+ 2y^6 - a^3b^3
\end{multline*}

\begin{myquote}
	\begin{lstlisting}
\begin{multline*}
p(x) = 3x^6 + 14x^5y\\
+ 590x^4y^2 + 19x^3y^3\\
- 12x^2y^4 - 12xy^5\\
+ 2y^6 - a^3b^3
\end{multline*}
	\end{lstlisting}
\end{myquote}

\subsubsection{Entorno \texttt{align}}

Este entorno se encarga de alinear las ecuaciones que estén escritas donde de ella. Este entorno funciona muy similar al entorno \texttt{split}, salvo que este no debe ser hecho dentro de un entorno \texttt{equation} y permite enumerar varias ecuaciones. Además, este formato permite alinear varias ecuaciones en columnas diferente. Finalmente cuenta con su versión sin enumeración \texttt{aling*}.

Las ecuaciones se deben separar con un \verb|\\| y los caracteres escogidos para alinear las ecuaciones deben ser acompañados de un \verb|&|:

\textbf{Ejemplo 1}

\begin{align}
y &= mx+b\\
mx+b &= y\\
x &= -b\pm \frac{\sqrt{b^2-4ac}}{2a}
\end{align}

\begin{myquote}
	\begin{lstlisting}
\begin{align}
	y &= mx+b\\
	mx+b &= y\\
	x &= -b\pm \frac{\sqrt{b^2-4ac}}{2a}
\end{align}
	\end{lstlisting}
\end{myquote}

\textbf{Ejemplo 2}

\begin{align}
y &= mx+b 	& 	a &= \pi \cdot r^2\\
mx+b &= y 	& 	\frac{b\cdot h}{2} &= a\\
x &= -b\pm \frac{\sqrt{b^2-4ac}}{2a} & h^2 &= c_1^2+c_2^2
\end{align}

\begin{myquote}
	\begin{lstlisting}
\begin{align}
	y &= mx+b 	& 	a &= \pi \cdot r^2\\
	mx+b &= y 	& 	\frac{b\cdot h}{2} &= a\\
	x &= -b\pm \frac{\sqrt{b^2-4ac}}{2a} & h^2 &= c_1^2+c_2^2
\end{align}
	\end{lstlisting}
\end{myquote}

\subsubsection{Entorno \texttt{gather}}

Este entorno nos permite centrar una serie de ecuaciones, sin alineación alguna de cierto carácter. Las ecuaciones deben ser separadas con un \verb|\\|. De igual manera al adicionar un asterisco se va a dejar de numerar las ecuaciones.

\textbf{Ejemplo 1}

\begin{gather*}
	y=mx+b\\
	mx+b=y\\
	x=-b\pm \frac{\sqrt{b^2-4ac}}{2a}
\end{gather*}

\begin{myquote}
	\begin{lstlisting}
\begin{gather*}
	y=mx+b\\
	mx+b=y\\
	x=-b\pm \frac{\sqrt{b^2-4ac}}{2a}
\end{gather*}
	\end{lstlisting}
\end{myquote}

\subsection{Paquete \texttt{amssymb}}

Agrega símbolos extra y dos tipos de fuentes matemáticas. Las dos fuentes nuevas son \verb|\mathfrak{}|, que es fuente Franktur y \verb|\mathbb{}| que es negrita de pizarra. De igual manera agrega muchos símbolos que pueden ser consultados en la siguiente página: \href{http://milde.users.sourceforge.net/LUCR/Math/mathpackages/amssymb-symbols.pdf}{\textcolor{light-blue}{Günter Milde}}

\textbf{Ejemplo 1}

\begin{gather*}
	\mathfrak{F}\\
	\mathbb{R}\\
	\curlyeqsucc	
\end{gather*}

\begin{myquote}
	\begin{lstlisting}
\begin{gather*}
	\mathfrak{F}\\
	\mathbb{R}\\
	\curlyeqsucc	
\end{gather*}
	\end{lstlisting}
\end{myquote}

\subsection{Matrices}

Hay que recalcar que para escribir matrices es necesario el uso del paquete \texttt{amsmath}. Para escribir una matriz debemos usar en entorno \texttt{matrix}, y al igual que las tablas, separamos las columnas con el signo \verb*|&| y se da un salto de fila con un signo \verb*|\\|. Este entorno debe de usarse dentro de un entorno matemático, como \texttt{equation}:

\textbf{Ejemplo 1}

\begin{equation*}
	\begin{matrix}
		1 & 2\\
		3 & 4
	\end{matrix}	
\end{equation*}

\begin{myquote}
	\begin{lstlisting}
\begin{equation*}
	\begin{matrix}
		1 & 2\\
		3 & 4
	\end{matrix}	
\end{equation*}
	\end{lstlisting}
\end{myquote}

\subsubsection{Delimitantes personalizados}

Para definir los delimitantes tenemos diferentes opciones. Lo que se debe hacer es cambiar el entorno, manteniendo igual el contenido de este:

\begin{center}
	\begin{tabularx}{0.8\textwidth}{|c|X|X|}
		\hline
		Tipo & Entorno & Ejemplo\\
		\hline
		Plano & \verb|\begin{matrix}| & $\begin{matrix} 1 & 2\\3 & 4\end{matrix}$\\
		&&\\
		Paréntesis & \verb|\begin{pmatrix}| & $\begin{pmatrix} 1 & 2\\3 & 4\end{pmatrix}$\\
		&&\\
		Paréntesis cuadrados & \verb|\begin{bmatrix}| & $\begin{bmatrix} 1 & 2\\3 & 4\end{bmatrix}$\\
		&&\\
		Corchetes & \verb|\begin{Bmatrix}| & $\begin{Bmatrix} 1 & 2\\3 & 4\end{Bmatrix}$\\
		&&\\
		Líneas & \verb|\begin{vmatrix}| & $\begin{vmatrix} 1 & 2\\3 & 4\end{vmatrix}$\\
		&&\\
		Líneas dobles & \verb|\begin{Vmatrix}| & $\begin{Vmatrix} 1 & 2\\3 & 4\end{Vmatrix}$\\
		&&\\
		\hline
	\end{tabularx}
\end{center}

\subsubsection{Entorno \texttt{smallmatrix} (Matrices en línea)}

Si se requiere insertar una matriz en la línea de texto se debe usar el entorno \texttt{smallmatrix} dentro del entorno matemático. Además, si se quiere modificar sus delimitantes se deberá usar \verb|\bigl| y \verb|\bigr| para establecer los determinantes. Veamos algunos ejemplos:

\textbf{Ejemplo 1}

Texto Texto Texto Texto Texto Texto \( \begin{smallmatrix} 1 & 2\\ 3 & 4 \end{smallmatrix}\) Texto Texto Texto Texto Texto Texto

\begin{myquote}
	\begin{lstlisting}
Texto Texto Texto Texto Texto Texto \( \begin{smallmatrix} 1 & 2\\ 3 & 4 \end{smallmatrix}\) Texto Texto Texto Texto Texto Texto
	\end{lstlisting}
\end{myquote}

\textbf{Ejemplo 2}

Texto Texto Texto Texto Texto Texto \( \bigl( \begin{smallmatrix} 1 & 2\\ 3 & 4 \end{smallmatrix} \bigr) \) Texto Texto Texto Texto Texto Texto

\begin{myquote}
	\begin{lstlisting}
Texto Texto Texto Texto Texto Texto \( \bigl( \begin{smallmatrix} 1 & 2\\ 3 & 4 \end{smallmatrix} \bigr) \) Texto Texto Texto Texto Texto Texto
	\end{lstlisting}
\end{myquote}

\textbf{Ejemplo 3}

Texto Texto Texto Texto Texto Texto \( \bigl\{ \begin{smallmatrix} 1 & 2\\ 3 & 4 \end{smallmatrix} \bigr\} \) Texto Texto Texto Texto Texto Texto

\begin{myquote}
	\begin{lstlisting}
Texto Texto Texto Texto Texto Texto \( \bigl\{ \begin{smallmatrix} 1 & 2\\ 3 & 4 \end{smallmatrix} \bigr\} \) Texto Texto Texto Texto Texto Texto
	\end{lstlisting}
\end{myquote}

\subsection{Creación de símbolos con comandos}

Muchas veces necesitamos utilizar símbolos que no tienen un comando asignado por defecto, por lo que los ponemos utilizando diferentes métodos, como tipos de letra, entre otros. Para acortar el tiempo que duramos al poner un símbolo complicado, vamos a crear un comando, o macro, para facilitar esta tarea.

La forma de crear un comando es utilizando \verb|\newcommand{comando}[argumentos][defecto]{definicion}|, esto se debe de posicionar en el preámbulo del documento. En la sección de \verb|comando| se va a definir el nombre que deseamos que tenga nuestro comando. \verb|argumento| es el número de argumentos que le podremos poner al comando de 1 a nueve, como vemos esta es opcional. \verb|defecto| es el argumento que viene por defecto si no se establece un argumento. Y finalmente, \verb|definicion| es donde escribimos lo que hace el comando. Para entender mejor esto veamos varios ejemplos: 

\textbf{Ejemplo 1}

\begin{myquote}
	\begin{lstlisting}
%Preambulo
\newcommand{\Lagr}{\mathcal{L}}		
	\end{lstlisting}
\end{myquote}

Sin comando: $\mathcal{L}$

Con comando: $\Lagr$

\begin{myquote}
	\begin{lstlisting}
Sin comando: $\mathcal{L}$
		
Con comando: $\Lagr$
	\end{lstlisting}
\end{myquote} 

Como vemos en este ejemplo, se creó el comando \verb|\Lagr| que permite hacer el símbolo de Lagrange sin la necesidad de escribir el comando entero.

\textbf{Ejemplo 2}

\begin{myquote}
	\begin{lstlisting}
%Preambulo
\newcommand{\R}[1][2]{\mathbb{R}^{#1}}
\usepackage{amssymb} %Se utiliza este paquete porque incluye el tipo de letra que queremos, es decir, \mathbb{}
	\end{lstlisting}
\end{myquote}

Sin comando: $\mathbb{R}^2$

Con comando: $\R$

\begin{myquote}
	\begin{lstlisting}
Sin comando: $\mathbb{R}^2$

Con comando: $\R$
	\end{lstlisting}
\end{myquote}

Como vemos en el ejemplo creamos el comando \verb|\R| que nos permite ilustrar un espacio bidimensional. Como no se especificó un argumento tomó el argumento por defecto, que en este caso es 2. Veamos otro ejemplo donde se especifique el argumento:

\textbf{Ejemplo 3}

\begin{myquote}
	\begin{lstlisting}
%Preambulo
\newcommand{\R}[1][2]{\mathbb{R}^{#1}}
\usepackage{amssymb} %Se utiliza este paquete porque incluye el tipo de letra que queremos, es decir, \mathbb{}		
	\end{lstlisting}
\end{myquote}

Sin comando: $\mathbb{R}^3$

Con comando: $\R[3]$

\begin{myquote}
	\begin{lstlisting}
Sin comando: $\mathbb{R}^3$

Con comando: $\R[3]$
	\end{lstlisting}
\end{myquote} 

Como vemos en este ejemplo, se especificó como argumento el número tres, lo que da el resultado del símbolo para los espacios tridimensionales.

\textbf{Ejemplo 4}

\begin{myquote}
	\begin{lstlisting}
%Preambulo
\newcommand{\Cels}{^{\circ}\text{C}}
	\end{lstlisting}
\end{myquote}

Sin comando: $^{\circ}\text{C}$

Con comando: $\Cels$

\begin{myquote}
	\begin{lstlisting}
Sin comando: $^{\circ}\text{C}$

Con comando: $\Cels$
	\end{lstlisting}
\end{myquote} 

En este ejemplo podemos ver cómo se agrega un comando para escribir el símbolo de los grados Celsius en los entornos matemáticos.

