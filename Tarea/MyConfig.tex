%\ProvidesPackage{MyConfig}

\usepackage[utf8]{inputenc }% Caracteres especiales
\usepackage[T1]{fontenc}
\usepackage[spanish]{babel} % Idioma del documento
\usepackage[letterpaper,top=2.5cm,bottom=2.5cm,left=2.5cm,right=2.5cm]{geometry} % Márgenes
\usepackage{graphicx} % Para insertar imágenes
\usepackage{float} % Permite posicionar mejor las figuras y tablas
\usepackage{ragged2e} % Para centrar y justificar
\usepackage{url} % Para poner URL
\usepackage{apacite} % Citar en estilo APA
\usepackage{natbib} % Para aumentar los comandos a la hora de citar 
\usepackage{enumerate} % Modificar las enumeraciones
\usepackage{multirow} % Para unir columnas y filas
\usepackage{tabularx} % Para hacer tablas que completen el largo de la página
\usepackage{array} % Según la wiki de tabular x, este paquete es necesario para que tabularx funcione bien. Además, agrega nuevas opciones a tabular
\usepackage{lmodern} % Cambia la fuente a una fuente vectorial, para que no haya problemas al reescalado
\usepackage{setspace} % Para establecer el interlineado
\usepackage{amsmath} % Mejora la escritura matemática
\usepackage{amsthm} % Ayuda a definir estructuras similares a teoremas
\usepackage{amssymb} % Añade símbolos matemáticos extra
\usepackage{titlesec} % Modificar los estidos de las secciones, subsecciones y subsubsecciones
%\usepackage{colortbl} % Color a las celdas de las tablas
%\usepackage{xcolor} % Poder definir colores
%\usepackage{longtable} % Para hacer tablas largas 
%\usepackage{slashbox,pict2e} % Para dar un slash en las tablas. pict2e mejora la línea
%\usepackage{cancel} % Poder cancelar en las ecuaciones matemáticas
%\usepackage{multicol} % Añade distribución para hacer que todo el documento pueda ser a dos columnas.
%\usepackage{caption} % Modifica el estilo del caption
\usepackage[hidelinks, breaklinks=true]{hyperref} % Linkear las citas y los urls. Hidelinks elimina los cuadros que se hacen a los lados. breaklinks=true se usa para que separe en líneas lo títulos muy largos

%%%%%%%%%%%%%%%%%%%%%%%%%%%%%%%%%%%%% Modificación secciones

\titleformat*{\section}{\large \bfseries \centering} % Da formato a \section
\titlespacing{\section}{0cm}{0cm}{0cm} % Espacios de las secciones. El primero indica el espacio entre el borde del documento y el número. El segundo el espacio con el párrafo de arriba y el tercero el espacio con el párrafo de abajo
%\titleformat*{\subsection}{\normalsize \bfseries \centering}
%\titleformat*{\subsubsection}{\normalsize \bfseries \centering}

%%%%%%%%%%%%%%%%%%%%%%%%%%%%%%%%%%%%% Modificación caption

%\captionsetup{font=small,justification=raggedright,singlelinecheck=false} % Modifica el estilo de los caption
%\captionsetup[figure]{name={Fig.},labelsep=period,labelfont={bf,it},textfont=it} % Modifica el nombre de las figuras y su separador
%\captionsetup[table]{name={Tabla},labelsep=newline,labelfont={bf}} % Modifica el nombre de las figuras y su separador

%%%%%%%%%%%%%%%%%%%%%%%%%%%%%%%%%%%%% Entornos tablas

%\renewcommand\tabularxcolumn[1]{>{\Centering}p{#1}} % Hace que el X de tabularx ahora esté centrado y no alineado a la izquierda
\newcolumntype{Y}{>{\raggedright\arraybackslash}X} % Crea el tipo Y que es como el X pero alineado a la izquierda

%%%%%%%%%%%%%%%%%%%%%%%%%%%%%%%%%%%%% Nuevos comandos

%\newcommand{\R}[1][2]{\mathbb{R}^{#1}} % Comando para espeficifar espacios dimencionales
%\newcommand{\Lagr}{\mathcal{L}} % Comando para escribir el símbolo de Lagrange 
%\newcommand{\Cels}{^{\circ}\text{C}} % Comando para escribir el símbolo de Celsius
%\newcommand{\U}{\mathcal{U}} % Comando para escribir el símbolo de Utilidad

%%%%%%%%%%%%%%%%%%%%%%%%%%%%%%%%%%%%% Nuevos entornos

%\newenvironment{myquote} % Para que lo que esté en el entrorno myquote tenga sangría de 0.5in
%{\list{}{\leftmargin=0.5in}\item[]}
%{\endlist} 

%%%%%%%%%%%%%%%%%%%%%%%%%%%%%%%%%%%%%