\usepackage[utf8]{inputenc}% Caracteres especiales
\usepackage[T1]{fontenc}
\usepackage[spanish]{babel} % Idioma del documento
\usepackage[letterpaper,top=1.5cm,bottom=2cm,left=1.5cm,right=1.5cm]{geometry} % Márgenes
\usepackage{graphicx} % Para insertar imágenes
\usepackage{float} % Permite posicionar mejor las figuras y tablas
\usepackage{ragged2e} % Para centrar y justificar
\usepackage{url} % Para poner URL 
\usepackage{enumerate} % Modificar las enumeraciones
\usepackage{multirow} % Para unir columnas y filas
\usepackage{colortbl} % Color a las celdas de las tablas
\usepackage[dvipsnames]{xcolor} % Poder definir colores. El [dvipsnames] agrega nombres de colores
\usepackage{longtable} % Para hacer tablas largas
\usepackage{tabularx} % Para hacer tablas que completen el largo de la página
\usepackage{array} % Según la wiki de tabular x, este paquete es necesario para que tabularx funcione bien. Además, agrega nuevas opciones a tabular
%\usepackage{slashbox,pict2e} % Para dar un slash en las tablas. pict2e mejora la línea
\usepackage{lmodern} % Cambia la fuente a una fuente vectorial, para que no haya problemas al reescalado
\usepackage{cancel} % Poder cancelar en las ecuaciones matemáticas
\usepackage{setspace} % Para establecer el interlineado
\usepackage{amsmath} % Mejora la escritura matemática
\usepackage{amsthm} % Ayuda a definir estructuras similares a teoremas
\usepackage{amssymb} % Añade símbolos matemáticos extra
\usepackage{titlesec} % Modificar los estilos de las secciones, subsecciones y subsubsecciones
\usepackage{multicol} % Añade distribución para hacer que todo el documento pueda ser a dos columnas.
\usepackage{newfloat} % Crear nuevos entornos flotantes
\usepackage{caption} % Modifica el estilo del caption
\usepackage[hidelinks, breaklinks=true]{hyperref} % Linkear las citas y los urls. Hidelinks elimina los cuadros que se hacen a los lados. breaklinks=true se usa para que separe en líneas lo títulos muy largos
\usepackage{booktabs} % Nuevas líneas para hacer tablas
\usepackage{fancyhdr} % Para modificar encabezados y pies de página

%%%%%%%%%%%%%%%%%%%%%%%%%%%%%%%%%%%%% Definir colores

\definecolor{Micolor}{rgb}{0.0, 0.5, 0.0} % Ao (English) (green)
%\definecolor{Micolor}{rgb}{0.54, 0.17, 0.89} % Blue violet
%\definecolor{Micolor}{rgb}{0.0, 0.44, 1.0} % Brandeis blue
%\definecolor{Micolor}{rgb}{0.8, 0.34, 0.0} % Tenné (Tawny)
%\definecolor{Micolor}{rgb}{0.68, 0.09, 0.13} % Upsdell red

%%%%%%%%%%%%%%%%%%%%%%%%%%%%%%%%%%%%% Modificación secciones

\titleformat{\section}[block]{\normalfont \large \bfseries \color{Micolor}}{\arabic{section}.}{4pt}{} % Da formato a \section
\titlespacing{\section}{0cm}{16pt}{10pt} % Espacios de las secciones. El primero indica el espacio entre el borde del documento y el número. El segundo el espacio con el párrafo de arriba y el tercero el espacio con el párrafo de abajo

\titleformat{\subsection}[block]{\normalfont \large \itshape \color{Micolor}}{\arabic{section}.\arabic{subsection}.}{4pt}{}
\titlespacing{\subsection}{0cm}{16pt}{10pt}

\titleformat*{\subparagraph}{\normalfont \normalsize \itshape \color{Micolor}}

%%%%%%%%%%%%%%%%%%%%%%%%%%%%%%%%%%%%% Nuevos entornos

\DeclareFloatingEnvironment[fileext=frm,placement={H}]{imagen} % Entorno imágenes
\DeclareFloatingEnvironment[fileext=frm,placement={H}]{esquema} % Entorno esquema

%%%%%%%%%%%%%%%%%%%%%%%%%%%%%%%%%%%%% Modificación caption

\captionsetup{font=small,justification=raggedright,singlelinecheck=false} % Modifica el estilo de los caption
\captionsetup[figure]{name={Fig.},labelsep=period,labelfont={bf,it,color=Micolor},textfont=it} % Modifica el nombre de las figuras y su separador
\captionsetup[imagen]{name={Img.},labelsep=period,labelfont={bf,it,color=Micolor},textfont=it} % Modifica el nombre de las imágenes y su separador
\captionsetup[esquema]{name={Esq.},labelsep=period,labelfont={bf,it,color=Micolor},textfont=it} % Modifica el nombre de los esquemas y su separador
\captionsetup[table]{name={Tabla},labelsep=newline,labelfont={bf,color=Micolor}} % Modifica el nombre de las tablas y su separador

%%%%%%%%%%%%%%%%%%%%%%%%%%%%%%%%%%%%% Entornos tablas

%\renewcommand\tabularxcolumn[1]{>{\Centering}p{#1}} % Hace que el X de tabularx ahora esté centrado y no alineado a la izquierda
\newcolumntype{Y}{>{\raggedright\arraybackslash}X} % Crea el tipo Y que es como el X pero alineado a la izquierda

%%%%%%%%%%%%%%%%%%%%%%%%%%%%%%%%%%%%% Configuración encabezado y pie de página

\pagestyle{fancy} % Estilo encabezado y pie de página personalizado

\fancyhf{} % Borrar el estilo predeterminado
\renewcommand{\headrulewidth}{0pt} % Grosor línea superior
\renewcommand{\footrulewidth}{0pt} % Grosor línea inferior

%\lhead{} % Superior izquierdo
%\chead{} % Superior centro
%\rhead{} % Superior derecha

%\lfoot{} % Inferior izquierdo
\cfoot{\textcolor{Micolor}{\rulefillfoot\;[\thepage]\;\rulefillfoot}} % Inferior centro
%\rfoot{} % Inferior derecha 

%%%%%%%%%%%%%%%%%%%%%%%%%%%%%%%%%%%%% Nuevos comandos

\newcommand{\rulefillfoot}{\leavevmode\leaders\hrule height .7ex width 1pt depth -0.6ex\hfill\kern0pt} % Se define un nuevo comando como \hrulefill pero a mitad del renglón
\newcommand{\rulefillimg}{\leavevmode\leaders\hrule height 0.6pt\hfill\kern0pt} % Línea para las imágenes
\newcommand{\linhor}{\noindent \textcolor{Micolor}{\rulefillimg}}

\newcommand{\cels}{$^{\circ}\text{C}$ }
\newcommand{\csf}{Diagrama elavorado empleando ChemSketch Freeware 2021.}

%%%%%%%%%%%%%%%%%%%%%%%%%%%%%%%%%%%%%